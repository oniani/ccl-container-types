%%%%%%%%%%%%%%%%%%%%%%%%%%%%%%%%%%%%%%%%%%%%%%%%%%%%%%%%%%%%%%%%%%%%%%%%%%%%%%%
%
% Filename: ccl-rules.tex
% Author:   David Oniani
% Modified: November 23, 2019
%  _         _____   __  __
% | |    __ |_   _|__\ \/ /
% | |   / _` || |/ _ \\  /
% | |__| (_| || |  __//  \
% |_____\__,_||_|\___/_/\_\
%
%%%%%%%%%%%%%%%%%%%%%%%%%%%%%%%%%%%%%%%%%%%%%%%%%%%%%%%%%%%%%%%%%%%%%%%%%%%%%%%

%%%%%%%%%%%%%%%%%%%%%%%%%%%%%%%%%%%%%%%%%%%%%%%%%%%%%%%%%%%%%%%%%%%%%%%%%%%%%%%
% Document Definition
%%%%%%%%%%%%%%%%%%%%%%%%%%%%%%%%%%%%%%%%%%%%%%%%%%%%%%%%%%%%%%%%%%%%%%%%%%%%%%%

\documentclass[11pt]{article}

%%%%%%%%%%%%%%%%%%%%%%%%%%%%%%%%%%%%%%%%%%%%%%%%%%%%%%%%%%%%%%%%%%%%%%%%%%%%%%%
% Packages
%%%%%%%%%%%%%%%%%%%%%%%%%%%%%%%%%%%%%%%%%%%%%%%%%%%%%%%%%%%%%%%%%%%%%%%%%%%%%%%

\usepackage[margin=1in]{geometry}
\usepackage[utf8]{inputenc}
\usepackage[english]{babel}
\usepackage{amsmath}
\usepackage{mathpartir}
\usepackage{fancyhdr}
\usepackage{tikz}
\usepackage{hyperref}

%%%%%%%%%%%%%%%%%%%%%%%%%%%%%%%%%%%%%%%%%%%%%%%%%%%%%%%%%%%%%%%%%%%%%%%%%%%%%%%
% Miscellaneous
%%%%%%%%%%%%%%%%%%%%%%%%%%%%%%%%%%%%%%%%%%%%%%%%%%%%%%%%%%%%%%%%%%%%%%%%%%%%%%%

% Some Setup
\pagestyle{fancy}
\setlength{\jot}{11pt}
\setlength{\parindent}{0pt}
\newcommand\rtypeof{\mathrel{:}}
\newcommand\ltypeof{\mathrel{::}}
\newcommand\subtype{\mathrel{<:}}
\newcommand\subtypecont{\mathrel{<::}}
\newcommand{\syntax}{\texttt}
\DeclareMathAlphabet{\syntaxvar}{OT1}{cmss}{m}{n}
\DeclareMathAlphabet{\type}{OT1}{pzc}{m}{it}

\newcommand\PlaceText[3]{%
\begin{tikzpicture}[remember picture,overlay]
\node[outer sep=0pt,inner sep=0pt,anchor=south west]
  at ([xshift=#1,yshift=-#2]current page.north west) {#3};
\end{tikzpicture}%
}

% PDF information and nice-looking urls
\hypersetup{%
  pdfauthor={David Oniani},
  pdftitle={Type Inference Rules For Container Types in CCL},
  pdfsubject={Programming Languages, Type Theory, Type Inference}
  pdfkeywords={Programming, Languages, Types},
  pdflang={English},
  colorlinks=true,
  linkcolor={black!50!blue},
  citecolor={black!50!blue},
  urlcolor={black!50!blue}
}

% Put a centered header of a footnote size on the top of each page
\chead{Type Inference Rules For Container Types in CCL}

%%%%%%%%%%%%%%%%%%%%%%%%%%%%%%%%%%%%%%%%%%%%%%%%%%%%%%%%%%%%%%%%%%%%%%%%%%%%%%%
% Beginning of the Document
%%%%%%%%%%%%%%%%%%%%%%%%%%%%%%%%%%%%%%%%%%%%%%%%%%%%%%%%%%%%%%%%%%%%%%%%%%%%%%%

\begin{document}

%%%%%%%%%%%%%%%%%%%%%%%%%%%%%%%%%%%%%%%%%%%%%%%%%%%%%%%%%%%%%%%%%%%%%%%%%%%%%%%
% Tikz
%%%%%%%%%%%%%%%%%%%%%%%%%%%%%%%%%%%%%%%%%%%%%%%%%%%%%%%%%%%%%%%%%%%%%%%%%%%%%%%

\PlaceText{30mm}{70mm}{\large Fonts}
\PlaceText{30mm}{73mm}{--------------------------------}
\PlaceText{30mm}{78mm}{\syntax{X,x} - Concrete syntax}
\PlaceText{30mm}{83mm}{$\syntaxvar{X,x}$ - Syntax variables}
\PlaceText{30mm}{88mm}{$\type{X,x}$ - Type variables}
\PlaceText{30mm}{92mm}{--------------------------------}

%%%%%%%%%%%%%%%%%%%%%%%%%%%%%%%%%%%%%%%%%%%%%%%%%%%%%%%%%%%%%%%%%%%%%%%%%%%%%%%
% The : and ::: Relation
%%%%%%%%%%%%%%%%%%%%%%%%%%%%%%%%%%%%%%%%%%%%%%%%%%%%%%%%%%%%%%%%%%%%%%%%%%%%%%%

\section*{The $\rtypeof$ and $\ltypeof$ Relations}

\begin{gather}
  \inferrule{\syntax{triv $\syntaxvar{x}$}}{\syntaxvar{x} \rtypeof \type{Triv}}\\
  \inferrule{\syntax{triv $\syntaxvar{x}$}}{\syntaxvar{x} \ltypeof \type{Box(Triv)}}\\
  \inferrule{\syntax{int $\syntaxvar{x}$}}{\syntaxvar{x} \rtypeof \type{Int}}\\
  \inferrule{\syntax{int $\syntaxvar{x}$}}{\syntaxvar{x} \ltypeof \type{Box(Int)}}\\
  \inferrule{\syntaxvar{x} \ltypeof \type{Box(T)}}{\syntax{immut}\ \syntaxvar{x} \ltypeof \type{IBox(T)}}\\
  \inferrule{\syntaxvar{x} \ltypeof \type{T}}{\syntax{ref}\ \syntaxvar{x} \ltypeof \type{Box(Ref(T))}}\\
  \inferrule{\syntaxvar{x} \ltypeof \type{T}}{\syntax{\&}\ \syntaxvar{x} \rtypeof \type{Ref(T)}}\\
  \inferrule{\syntaxvar{x} \rtypeof \type{Ref(Box(T))}}{\syntaxvar{x}\ \syntax{@} \rtypeof \type{T}}\\
  \inferrule{\syntaxvar{x} \rtypeof \type{Ref(IBox(T))}}{\syntaxvar{x}\ \syntax{@} \rtypeof \type{T}}\\
  \inferrule{\syntaxvar{x} \rtypeof \type{Ref(T)}}{\syntaxvar{x}\ \syntax{@} \ltypeof \type{T}}\\
  \inferrule{\syntaxvar{x} \ltypeof \type{T} \\ \syntaxvar{y} \ltypeof \type{U} \\ \type{T} \subtypecont \type{U} \\ \syntaxvar{x} \rtypeof \type{V}}{\syntaxvar{x}\ \syntax{:=}\ \syntaxvar{y} \rtypeof \type{V}}
\end{gather}

$\syntax{x} : \type{T}$ is read as ``expression \syntax{x} is of type
$\type{T}$ and is in an \textit{r-context}.''

$\syntax{x} :: \type{T}$ is read as ``variable \syntax{x} is of type $\type{T}$
and is in an \textit{l-context}.''

The operators could also be referred to as the ``r-type of'' and ``l-type of''
operators.

\medskip

l-context denotes everything that is \textit{assignable} (indicated as a
storable memory).\ r-context, on the other hand, denotes everything that is
\textit{expressible} (can be produced by an expression).

\medskip

There is no r-value (e.g.\ expression) of the type $\type{Box(T)}$.

\medskip

For now, we omit rules for $\type{Con}$ types as they only operate on r-values.

\medskip

For now, we omit rules for $\type{Fun}$ types as they only accept r-values.
Any variable and/or primitive type has both r-value and l-value (when it comes
to primitive types, only r-value). In all cases, the r-value part of the actual
parameter is passed when the function is being called.

%%%%%%%%%%%%%%%%%%%%%%%%%%%%%%%%%%%%%%%%%%%%%%%%%%%%%%%%%%%%%%%%%%%%%%%%%%%%%%%
% The \subtypecont Relation
%%%%%%%%%%%%%%%%%%%%%%%%%%%%%%%%%%%%%%%%%%%%%%%%%%%%%%%%%%%%%%%%%%%%%%%%%%%%%%%

\section*{The $\subtypecont$ Relation}

\begin{gather}
  \inferrule{\type{}}{\type{Box(Triv)} \subtypecont \type{Box(Triv)}}\\
  \inferrule{\type{}}{\type{Box(Int)} \subtypecont \type{Box(Int)}}\\
  \inferrule{\type{T} \subtypecont \type{Box(U)}}{\type{T} \subtypecont \type{IBox(U)}}\\
  \inferrule{\type{Box(T)} \subtypecont \type{IBox(U)}}{\type{IBox(T)} \subtypecont \type{IBox(U)}}\\
  \inferrule{\type{T} \subtypecont \type{U}}{\type{Ref\ T} \subtypecont \type{Ref\ U}}
\end{gather}

$\type{Box?}$ statement checks for both mutable and immutable containers.
In other words, $\type{Box(Int)}$ and $\type{Immut(Box(Int))}$ would both
satisfy the $\type{Box?}$ condition.

%%%%%%%%%%%%%%%%%%%%%%%%%%%%%%%%%%%%%%%%%%%%%%%%%%%%%%%%%%%%%%%%%%%%%%%%%%%%%%%
% The End of the Document
%%%%%%%%%%%%%%%%%%%%%%%%%%%%%%%%%%%%%%%%%%%%%%%%%%%%%%%%%%%%%%%%%%%%%%%%%%%%%%%

\end{document}
