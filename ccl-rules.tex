%%%%%%%%%%%%%%%%%%%%%%%%%%%%%%%%%%%%%%%%%%%%%%%%%%%%%%%%%%%%%%%%%%%%%%%%%%%%%%%
%
% Filename: ccl-rules.tex
% Author:   David Oniani
% Modified: November 02, 2019
%  _         _____   __  __
% | |    __ |_   _|__\ \/ /
% | |   / _` || |/ _ \\  /
% | |__| (_| || |  __//  \
% |_____\__,_||_|\___/_/\_\
%
%%%%%%%%%%%%%%%%%%%%%%%%%%%%%%%%%%%%%%%%%%%%%%%%%%%%%%%%%%%%%%%%%%%%%%%%%%%%%%%

%%%%%%%%%%%%%%%%%%%%%%%%%%%%%%%%%%%%%%%%%%%%%%%%%%%%%%%%%%%%%%%%%%%%%%%%%%%%%%%
% Document Definition
%%%%%%%%%%%%%%%%%%%%%%%%%%%%%%%%%%%%%%%%%%%%%%%%%%%%%%%%%%%%%%%%%%%%%%%%%%%%%%%

\documentclass{article}

%%%%%%%%%%%%%%%%%%%%%%%%%%%%%%%%%%%%%%%%%%%%%%%%%%%%%%%%%%%%%%%%%%%%%%%%%%%%%%%
% Packages
%%%%%%%%%%%%%%%%%%%%%%%%%%%%%%%%%%%%%%%%%%%%%%%%%%%%%%%%%%%%%%%%%%%%%%%%%%%%%%%

\usepackage[margin=1in]{geometry}
\usepackage[utf8]{inputenc}
\usepackage[english]{babel}
\usepackage{amsmath}
\usepackage{mathpartir}
\usepackage{fancyhdr}
\usepackage{xcolor}
\usepackage{hyperref}

%%%%%%%%%%%%%%%%%%%%%%%%%%%%%%%%%%%%%%%%%%%%%%%%%%%%%%%%%%%%%%%%%%%%%%%%%%%%%%%
% Miscellaneous
%%%%%%%%%%%%%%%%%%%%%%%%%%%%%%%%%%%%%%%%%%%%%%%%%%%%%%%%%%%%%%%%%%%%%%%%%%%%%%%

% Some Setup
\pagestyle{fancy}
\setlength{\jot}{11pt}
\setlength{\parindent}{0pt}
\newcommand\rtypeof{\mathrel{:}}
\newcommand\ltypeof{\mathrel{::}}
\newcommand\subtype{\mathrel{<:}}
\newcommand{\syntax}{\texttt}
\DeclareMathAlphabet{\type}{OT1}{pzc}{m}{it}

% PDF information and nice-looking urls
\hypersetup{%
  pdfauthor={David Oniani},
  pdftitle={Type Inference Rules For Container Types in CCL},
  pdfsubject={Programming Languages, Type Theory, Type Inference}
  pdfkeywords={Programming, Languages, Types},
  pdflang={English},
  colorlinks=true,
  linkcolor={black!50!blue},
  citecolor={black!50!blue},
  urlcolor={black!50!blue}
}

% Put a centered header of a footnote size on the top of each page
\chead{Type Inference Rules For Container Types in CCL}

%%%%%%%%%%%%%%%%%%%%%%%%%%%%%%%%%%%%%%%%%%%%%%%%%%%%%%%%%%%%%%%%%%%%%%%%%%%%%%%
% Author(s), Title, and Date
%%%%%%%%%%%%%%%%%%%%%%%%%%%%%%%%%%%%%%%%%%%%%%%%%%%%%%%%%%%%%%%%%%%%%%%%%%%%%%%

% Author(s)
\author{David Oniani\\
        Luther College\\
        \href{mailto:oniada01@luther.edu}{oniada01@luther.edu}}

% Title
\title{\textbf{Type Inference Rules For Container Types in CCL}\\
       \medskip
       \small \textsuperscript{*}The paper is written under the direction and
       guidance of Dr.\ Alan K. Zaring
       (\href{mailto:akzaring@luther.edu}{akzaring@luther.edu}).}

% Date
\date{Last Updated: November 02, 2019}

%%%%%%%%%%%%%%%%%%%%%%%%%%%%%%%%%%%%%%%%%%%%%%%%%%%%%%%%%%%%%%%%%%%%%%%%%%%%%%%
% Beginning of the Document
%%%%%%%%%%%%%%%%%%%%%%%%%%%%%%%%%%%%%%%%%%%%%%%%%%%%%%%%%%%%%%%%%%%%%%%%%%%%%%%

\begin{document}
\maketitle

%%%%%%%%%%%%%%%%%%%%%%%%%%%%%%%%%%%%%%%%%%%%%%%%%%%%%%%%%%%%%%%%%%%%%%%%%%%%%%%
% Abstract
%%%%%%%%%%%%%%%%%%%%%%%%%%%%%%%%%%%%%%%%%%%%%%%%%%%%%%%%%%%%%%%%%%%%%%%%%%%%%%%

\begin{abstract}
\noindent We present the type inference rules introducing the notion of
container types in the CCL programming language. This redesign required a
number of substantial changes to the major aspects of the language, including
the type system, the syntax, and the definitional semantics.
\end{abstract}

%%%%%%%%%%%%%%%%%%%%%%%%%%%%%%%%%%%%%%%%%%%%%%%%%%%%%%%%%%%%%%%%%%%%%%%%%%%%%%%
% The \subtype Relation
%%%%%%%%%%%%%%%%%%%%%%%%%%%%%%%%%%%%%%%%%%%%%%%%%%%%%%%%%%%%%%%%%%%%%%%%%%%%%%%

\section*{The $\subtype$ Relation}

\begin{gather}
  \inferrule{\type{}}{\type{Box(Triv)} \subtype \type{Box(Triv)}}\\
  \inferrule{\type{}}{\type{Box(Int)} \subtype \type{Box(Int)}}\\
  \inferrule{\type{T} \subtype \type{Box(U)} \\ \type{Box?(T)}}{\type{T} \subtype \type{Immut\ Box(U)}}\\
  \inferrule{\type{Box(T)} \subtype \type{Immut\ Box(U)}}{\type{immut\ Box(T)} \subtype \type{Immut\ Box(U)}}\\
  \inferrule{\type{T} \subtype \type{U} \\ \type{Box?(T)} \\ \type{Box?(U)}}{\type{Ref\ T} \subtype \type{Ref\ U}}
\end{gather}

$\type{Box?}$ statement checks for both mutable and immutable containers.
In other words, $\type{Box(Int)}$ and $\type{immut\ Box(Int)}$ would both
satisfy the $\type{Box?}$ condition.

%%%%%%%%%%%%%%%%%%%%%%%%%%%%%%%%%%%%%%%%%%%%%%%%%%%%%%%%%%%%%%%%%%%%%%%%%%%%%%%
% The : Relation
%%%%%%%%%%%%%%%%%%%%%%%%%%%%%%%%%%%%%%%%%%%%%%%%%%%%%%%%%%%%%%%%%%%%%%%%%%%%%%%

\section*{The $\rtypeof$ and $\ltypeof$ Relations}

\begin{gather}
  \inferrule{\syntax{triv x}}{\syntax{x} \rtypeof \type{Triv}}\\
  \inferrule{\syntax{triv x}}{\syntax{x} \ltypeof \type{Box(Triv)}}\\
  \inferrule{\syntax{int x}}{\syntax{x} \rtypeof \type{Int}}\\
  \inferrule{\syntax{int x}}{\syntax{x} \ltypeof \type{Box(Int)}}\\
  \inferrule{\syntax{x} \ltypeof \type{Box(T)}}{\syntax{immut\ x} \ltypeof \type{Immut\ Box(T)}}\\
  \inferrule{\syntax{x} \ltypeof \type{T} \\ \type{Box?(T)}}{\syntax{ref\ x} \ltypeof \type{Box(Ref\ T)}}\\
  \inferrule{\syntax{x} \ltypeof \type{T} \\ \type{Box?(T)}}{\syntax{(\ x\ )} \ltypeof \type{T}}\\
  \inferrule{\syntax{x} \ltypeof \type{T} \\ \type{Box?(T)}}{\syntax{\&\ x} \rtypeof \type{Ref\ T}}\\
  \inferrule{\syntax{x} \ltypeof \type{Ref\ T}}{\syntax{x\ @} \rtypeof \type{T}}\\
  \inferrule{\syntax{x} \rtypeof \type{Ref\ T}}{\syntax{x\ @} \rtypeof \type{T}}\\
  \inferrule{\syntax{x} \ltypeof \type{Box(T)} \\ \syntax{y} \ltypeof \type{U} \\ \type{T < : U}}{\syntax{x := y}:\type{Box(T)}}\\
  \inferrule{\syntax{x} \ltypeof \type{Box(T)} \\ \syntax{y} \ltypeof \type{U} \\ \type{T < : U} \\ \type{Box?(U)}}{\syntax{x := y}:\type{Box(T)}}
\end{gather}

$\syntax{x} : \type{T}$ is read as ``expression \syntax{x} is of type
$\type{T}$ and is in an \textit{r-context}.''

$\syntax{x} :: \type{T}$ is read as ``variable \syntax{x} is of type $\type{T}$
and is in an \textit{l-context}.''

The operators could also be referred to as the ``r-type of'' and ``l-type of''
operators.

\medskip

l-context denotes everything that is \textit{assignable} (indicated as a
storable memory).\ r-context, on the other hand, denotes everything that is
\textit{expressible} (can be produced by an expression).

\medskip

There is no r-value (e.g.\ expression) of the type $\type{Box(T)}$.

\medskip

We omit rules for $\type{Con}$ types as they only operate on r-values.

\medskip

We omit rules for $\type{Fun}$ types as they only accept r-values.
Any variable and/or primitive type has both r-value and l-value (when it comes
to primitive types, only r-value). In all cases, the r-value part of the actual
parameter is passed when the function is being called.

%%%%%%%%%%%%%%%%%%%%%%%%%%%%%%%%%%%%%%%%%%%%%%%%%%%%%%%%%%%%%%%%%%%%%%%%%%%%%%%
% Basic Rules
%%%%%%%%%%%%%%%%%%%%%%%%%%%%%%%%%%%%%%%%%%%%%%%%%%%%%%%%%%%%%%%%%%%%%%%%%%%%%%%

\section*{Resulting Relationships (A Short List)}

\begin{align*}
  &\syntax{int i} &&\to i &&\to \type{Box(Int)}\\
  &\syntax{immut int ii} &&\to ii &&\to \type{Immut\ Box(Int)}\\
  &\syntax{ref int ri} &&\to ri &&\to \type{Box(Ref\ Box(Int))}\\
  &\syntax{immut ref int iri} &&\to iri &&\to \type{Immut\ Box(Ref\ Box(Int))}\\
  &\syntax{ref immut int rii} &&\to rii &&\to \type{Box(Ref\ (Immut\ Box(Int)))}\\
  &\syntax{immut ref immut int irii} &&\to irii &&\to \type{Immut\ Box(Ref\ (Immut\ Box(Int)))}
\end{align*}

\medskip

Type $\type{Immut\ Box(Immut\ Box(Int))}$ cannot exist. Nested $\type{Box}$
types are only possible when there is at least one $\type{Ref}$ type.

\begin{gather*}
  \type{Box(Triv)} \subtype \type{Box(Triv)}\\
  \type{Box(Int)} \subtype \type{Box(Int)}\\
  \type{Box(Triv)} \subtype \type{Immut\ Box(Triv)}\\
  \type{Box(Int)} \subtype \type{Immut\ Box(Int)}\\
  \type{Immut\ Box(Triv)} \subtype \type{Immut\ Box(Triv)}\\
  \type{Immut\ Box(Int)} \subtype \type{Immut\ Box(Int)}
\end{gather*}

\medskip

All the rules above should work with $\type{Ref}$ types in the similar manner:

\begin{gather*}
  \type{Ref\ Box(Triv)} \subtype \type{Ref\ Box(Triv)}\\
  \type{Ref\ Box(Int)} \subtype \type{Ref\ Box(Int)}\\
  \type{Ref\ Box(Triv)} \subtype \type{Ref\ Immut\ Box(Triv)}\\
  \type{Ref\ Box(Int)} \subtype \type{Ref\ Immut\ Box(Int)}\\
  \type{Ref\ Immut\ Box(Triv)} \subtype \type{Ref\ Immut\ Box(Triv)}\\
  \type{Ref\ Immut\ Box(Int)} \subtype \type{Ref\ Immut\ Box(Int)}
\end{gather*}

%%%%%%%%%%%%%%%%%%%%%%%%%%%%%%%%%%%%%%%%%%%%%%%%%%%%%%%%%%%%%%%%%%%%%%%%%%%%%%%
% The End of the Document
%%%%%%%%%%%%%%%%%%%%%%%%%%%%%%%%%%%%%%%%%%%%%%%%%%%%%%%%%%%%%%%%%%%%%%%%%%%%%%%

\end{document}
