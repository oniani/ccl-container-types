%%%%%%%%%%%%%%%%%%%%%%%%%%%%%%%%%%%%%%%%%%%%%%%%%%%%%%%%%%%%%%%%%%%%%%%%%%%%%%%
%
% Filename: paper.tex
% Author:   David Oniani
% Modified: December 12, 2019
%  _         _____   __  __
% | |    __ |_   _|__\ \/ /
% | |   / _` || |/ _ \\  /
% | |__| (_| || |  __//  \
% |_____\__,_||_|\___/_/\_\
%
%%%%%%%%%%%%%%%%%%%%%%%%%%%%%%%%%%%%%%%%%%%%%%%%%%%%%%%%%%%%%%%%%%%%%%%%%%%%%%%

%%%%%%%%%%%%%%%%%%%%%%%%%%%%%%%%%%%%%%%%%%%%%%%%%%%%%%%%%%%%%%%%%%%%%%%%%%%%%%%
% Document Definition
%%%%%%%%%%%%%%%%%%%%%%%%%%%%%%%%%%%%%%%%%%%%%%%%%%%%%%%%%%%%%%%%%%%%%%%%%%%%%%%

\documentclass{article}

%%%%%%%%%%%%%%%%%%%%%%%%%%%%%%%%%%%%%%%%%%%%%%%%%%%%%%%%%%%%%%%%%%%%%%%%%%%%%%%
% Packages
%%%%%%%%%%%%%%%%%%%%%%%%%%%%%%%%%%%%%%%%%%%%%%%%%%%%%%%%%%%%%%%%%%%%%%%%%%%%%%%

\usepackage[margin=1in]{geometry}
\usepackage[utf8]{inputenc}
\usepackage[english]{babel}
\usepackage{amsmath}
\usepackage{mathpartir}
\usepackage{fancyhdr}
\usepackage{xcolor}
\usepackage{hyperref}

%%%%%%%%%%%%%%%%%%%%%%%%%%%%%%%%%%%%%%%%%%%%%%%%%%%%%%%%%%%%%%%%%%%%%%%%%%%%%%%
% Miscellaneous
%%%%%%%%%%%%%%%%%%%%%%%%%%%%%%%%%%%%%%%%%%%%%%%%%%%%%%%%%%%%%%%%%%%%%%%%%%%%%%%

% Some Setup
\pagestyle{fancy}
\setlength{\jot}{11pt}
\setlength{\parindent}{0pt}
\newcommand\rtypeof{\mathrel{:}}
\newcommand\ltypeof{\mathrel{::}}
\newcommand\subtype{\mathrel{<:}}
\newcommand\subtypecont{\mathrel{<::}}
\newcommand{\syntax}{\texttt}
\DeclareMathAlphabet{\syntaxvar}{OT1}{cmss}{m}{n}
\DeclareMathAlphabet{\type}{OT1}{pzc}{m}{it}

% PDF information and nice-looking urls
\hypersetup{%
  pdfauthor={David Oniani},
  pdftitle={Type Inference Rules For Container Types in CCL},
  pdfsubject={Programming Languages, Type Theory, Type Inference}
  pdfkeywords={Programming, Languages, Types},
  pdflang={English},
  colorlinks=true,
  linkcolor={black!50!blue},
  citecolor={black!50!blue},
  urlcolor={black!50!blue}
}

% Put a centered header of a footnote size on the top of each page
\chead{Type Inference Rules For Container Types in CCL}

%%%%%%%%%%%%%%%%%%%%%%%%%%%%%%%%%%%%%%%%%%%%%%%%%%%%%%%%%%%%%%%%%%%%%%%%%%%%%%%
% Author(s), Title, and Date
%%%%%%%%%%%%%%%%%%%%%%%%%%%%%%%%%%%%%%%%%%%%%%%%%%%%%%%%%%%%%%%%%%%%%%%%%%%%%%%

% Author(s)
\author{David Oniani\\
        Luther College\\
        \href{mailto:oniada01@luther.edu}{oniada01@luther.edu}}

% Title
\title{\textbf{Type Inference Rules For Container Types in CCL}\\
       \medskip
       \small \textsuperscript{*}The paper is written under the guidance of
       Dr.\ Alan K. Zaring
       (\href{mailto:akzaring@luther.edu}{akzaring@luther.edu}).}

% Date
\date{\today}

%%%%%%%%%%%%%%%%%%%%%%%%%%%%%%%%%%%%%%%%%%%%%%%%%%%%%%%%%%%%%%%%%%%%%%%%%%%%%%%
% Beginning of the Document
%%%%%%%%%%%%%%%%%%%%%%%%%%%%%%%%%%%%%%%%%%%%%%%%%%%%%%%%%%%%%%%%%%%%%%%%%%%%%%%

\begin{document}
\maketitle

%%%%%%%%%%%%%%%%%%%%%%%%%%%%%%%%%%%%%%%%%%%%%%%%%%%%%%%%%%%%%%%%%%%%%%%%%%%%%%%
% Abstract
%%%%%%%%%%%%%%%%%%%%%%%%%%%%%%%%%%%%%%%%%%%%%%%%%%%%%%%%%%%%%%%%%%%%%%%%%%%%%%%

\begin{abstract}
\noindent We present the type inference rules introducing the notion of
container types in the CCL programming language. This redesign required a
number of substantial changes to the major aspects of the type system and the
syntax of the language. Two new types $\type{Box}$ and $\type{IBox}$ were
introduced. In addition, two new operators $\subtypecont$ and $\ltypeof$ were
designed for dealing with the new types.
\end{abstract}

%%%%%%%%%%%%%%%%%%%%%%%%%%%%%%%%%%%%%%%%%%%%%%%%%%%%%%%%%%%%%%%%%%%%%%%%%%%%%%%
% The Box and IBox Types
%%%%%%%%%%%%%%%%%%%%%%%%%%%%%%%%%%%%%%%%%%%%%%%%%%%%%%%%%%%%%%%%%%%%%%%%%%%%%%%

\section*{The $\type{Box}$ and $\type{IBox}$ Types}

CCL is, by and large, value-based language. This makes it rather difficult to
have types more complex than what already comprises CCL. Besides, this is a
major obstacle for the implementation of heterogenous types such as
heterogenous ordered pair, heterogenous vector types, etc. One possible
solution to this problem is introducing the notion of container types.

\medskip

Container types are boxes which are either mutable or immutable. They wrap the
values allowing the composition of compound types by bringing the higher levels
of abstraction. Mutable and immutable types are represented by types
$\type{Box}$ and $\type{IBox}$ respectively. For some type $\type{T}$, the
result of the wrapping is either $\type{Box(T)}$ or $\type{IBox(T)}$.

%%%%%%%%%%%%%%%%%%%%%%%%%%%%%%%%%%%%%%%%%%%%%%%%%%%%%%%%%%%%%%%%%%%%%%%%%%%%%%%
% The <:: Relation
%%%%%%%%%%%%%%%%%%%%%%%%%%%%%%%%%%%%%%%%%%%%%%%%%%%%%%%%%%%%%%%%%%%%%%%%%%%%%%%

\section*{The $\subtypecont$ Relation}

For expressing relationships between container types, we introduce a new
$\subtypecont$ operator. The sole purpose of this operator is to specify the
subtype relation between two container types. As an example, $\type{T}
\subtypecont \type{U}$ is read as ``container type $\type{T}$ is a subtype of
the container type $\type{U}$.'' We proceed by introducing rules based on the
new operator.

\begin{gather}
  \inferrule{\type{}}{\type{Box(Triv)} \subtypecont \type{Box(Triv)}}\\
  \inferrule{\type{}}{\type{Box(Int)} \subtypecont \type{Box(Int)}}
\end{gather}

Rules 1 and 2 specify the reflexive property of the operator on types
$\type{Triv}$ and $\type{Int}$.

\begin{gather}
  \inferrule{\type{T} \subtypecont \type{Box(U)}}{\type{T} \subtypecont \type{IBox(U)}}\\
  \inferrule{\type{Box(T)} \subtypecont \type{IBox(U)}}{\type{IBox(T)} \subtypecont \type{IBox(U)}}\\
  \inferrule{\type{T} \subtypecont \type{U}}{\type{Ref\ T} \subtype \type{Ref\ U}}
\end{gather}

Rules 4 and 5 show the subtype relationship between container types and rule
6 describes the relationship between $\type{Ref}$ types.

%%%%%%%%%%%%%%%%%%%%%%%%%%%%%%%%%%%%%%%%%%%%%%%%%%%%%%%%%%%%%%%%%%%%%%%%%%%%%%%
% The :: Relation
%%%%%%%%%%%%%%%%%%%%%%%%%%%%%%%%%%%%%%%%%%%%%%%%%%%%%%%%%%%%%%%%%%%%%%%%%%%%%%%

\section*{The $\ltypeof$ Relation -- L-types and R-types}

In order to have a clear separation between containers and values, we introduce
a new operator $\ltypeof$ for specifying an l-type of variable/syntax variable.
$\syntax{x} :: \type{T}$ is read as ``variable \syntax{x} is of type $\type{T}$
and is in an \textit{l-context}.'' We call the already existing $\rtypeof$
operator the r-type operator. $\syntax{x} : \type{T}$ is read as ``expression
\syntax{x} is of type $\type{T}$ and is in an \textit{r-context}.'' The
operators could also be referred to as the ``r-type of'' and ``l-type of''
operators. It should be noted that l-context denotes everything that is
\textit{assignable} (indicated as a storable memory) and the r-context, on the
other hand, denotes everything that is \textit{expressible} (can be produced by
an expression). There is no r-value of the type $\type{Box(T)}$.

\begin{gather}
  \inferrule{\syntax{triv $\syntaxvar{x}$}}{\syntaxvar{x} \rtypeof \type{Triv}}\\
  \inferrule{\syntax{triv $\syntaxvar{x}$}}{\syntaxvar{x} \ltypeof \type{Box(Triv)}}\\
  \inferrule{\syntax{int $\syntaxvar{x}$}}{\syntaxvar{x} \rtypeof \type{Int}}\\
  \inferrule{\syntax{int $\syntaxvar{x}$}}{\syntaxvar{x} \ltypeof \type{Box(Int)}}
\end{gather}

Rules 6 and 7 show the l-value and r-value of the type $\type{triv}$ while
rules 8 and 9 specify similar relationships for the $\type{int}$ type.

\begin{gather}
  \inferrule{\syntaxvar{x} \ltypeof \type{Box(T)}}{\syntax{immut}\ \syntaxvar{x} \rtypeof \type{T}}
\end{gather}

Rule 10 shows the r-value of the immutable syntax variable.

\begin{gather}
  \inferrule{\syntaxvar{x} \ltypeof \type{T}}{\syntax{ref}\ \syntaxvar{x} \rtypeof \type{Ref(T)}}\\
  \inferrule{\syntaxvar{x} \ltypeof \type{T}}{\syntax{ref}\ \syntaxvar{x} \ltypeof \type{Box(Ref(T))}}
\end{gather}

Rules 11 and 12 describe the behavior of the \syntax{ref} operator when applied
on container types.

\begin{gather}
  \inferrule{\syntaxvar{x} \ltypeof \type{T}}{\syntax{\&}\ \syntaxvar{x} \rtypeof \type{Ref(T)}}
\end{gather}

Rule 13 specifies the behavior of \syntax{\&} operator, when applied on the
container type.

\begin{gather}
  \inferrule{\syntaxvar{x} \rtypeof \type{Ref(Box(T))}}{\syntaxvar{x}\ \syntax{@} \rtypeof \type{T}}\\
  \inferrule{\syntaxvar{x} \rtypeof \type{Ref(IBox(T))}}{\syntaxvar{x}\ \syntax{@} \rtypeof \type{T}}\\
  \inferrule{\syntaxvar{x} \rtypeof \type{Ref(T)}}{\syntaxvar{x}\ \syntax{@} \ltypeof \type{T}}
\end{gather}

Rules 14, 15, and 16 show the behavior of the \syntax{@} operator under
different conditions.

\begin{gather}
  \inferrule{\syntaxvar{x} \ltypeof \type{T} \\ \syntaxvar{y} \ltypeof \type{U} \\ \type{T} \subtypecont \type{U} \\ \syntaxvar{x} \rtypeof \type{V}}{\syntaxvar{x}\ \syntax{:=}\ \syntaxvar{y} \rtypeof \type{V}}
\end{gather}

Rule 17 shows the behavior of the assignment operator.
\newline

{\Large Notes}

For now, we omit rules for $\type{Con}$ types as they only operate on r-values.

\medskip

For now, we omit rules for $\type{Fun}$ types as they only accept r-values.
Any variable and/or primitive type has both r-value and l-value (when it comes
to primitive types, only r-value). In all cases, the r-value part of the actual
parameter is passed when the function is being called.

%%%%%%%%%%%%%%%%%%%%%%%%%%%%%%%%%%%%%%%%%%%%%%%%%%%%%%%%%%%%%%%%%%%%%%%%%%%%%%%
% Basic Rules
%%%%%%%%%%%%%%%%%%%%%%%%%%%%%%%%%%%%%%%%%%%%%%%%%%%%%%%%%%%%%%%%%%%%%%%%%%%%%%%

\section*{Example Relationships}

\begin{align*}
  &\syntax{int i}                    &&\to i    &&\to \type{Box(Int)}\\
  &\syntax{immut int ii}             &&\to ii   &&\to \type{IBox(Int)}\\
  &\syntax{ref int ri}               &&\to ri   &&\to \type{Box(Ref(Box(Int)))}\\
  &\syntax{immut ref int iri}        &&\to iri  &&\to \type{IBox(Ref(Box(Int)))}\\
  &\syntax{ref immut int rii}        &&\to rii  &&\to \type{Box(Ref((IBox(Int))))}\\
  &\syntax{immut ref immut int irii} &&\to irii &&\to \type{IBox(Ref((IBox(Int))))}
\end{align*}

\medskip

Type $\type{Immut\ Box(Immut\ Box(Int))}$ cannot exist. Nested $\type{Box}$
types are only possible when there is at least one $\type{Ref}$ type.

\begin{gather*}
  \type{Box(Triv)}  \subtypecont \type{Box(Triv)}\\
  \type{Box(Int)}   \subtypecont \type{Box(Int)}\\
  \type{Box(Triv)}  \subtypecont \type{IBox(Triv)}\\
  \type{Box(Int)}   \subtypecont \type{IBox(Int)}\\
  \type{IBox(Triv)} \subtypecont \type{IBox(Triv)}\\
  \type{IBox(Int)}  \subtypecont \type{IBox(Int)}
\end{gather*}

%%%%%%%%%%%%%%%%%%%%%%%%%%%%%%%%%%%%%%%%%%%%%%%%%%%%%%%%%%%%%%%%%%%%%%%%%%%%%%%
% The End of the Document
%%%%%%%%%%%%%%%%%%%%%%%%%%%%%%%%%%%%%%%%%%%%%%%%%%%%%%%%%%%%%%%%%%%%%%%%%%%%%%%

\end{document}
